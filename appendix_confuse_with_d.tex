\ESKDappendix{рекомендуемое}{Дополнительные пояснения к порядку расчета матрицы инерции}\label{app_confuse_with_d}
Рассмотрим следующие два математических выражения:
\begin{multline}\label{eq_confuse_with_d_example_1}
    \begin{bmatrix}
        a_1 & a_2 & a_3
    \end{bmatrix}
    \cdot
    \begin{bmatrix}
        b_{11} & b_{12} & b_{13} \\
        b_{21} & b_{22} & b_{23} \\
        b_{31} & b_{32} & b_{33} \\
    \end{bmatrix}
    \cdot
    \begin{bmatrix}
        a_1 \\ a_2 \\ a_3
    \end{bmatrix}
    = b_{11} a_1^2 + b_{22} a_2^2 + b_{33} a_3^2 + {} \\
    {} + (b_{12} + b_{21}) a_1 a_2 + (b_{13} + b_{31}) a_1 a_3 + (b_{23} + b_{32}) a_2 a_3,
\end{multline}

\begin{multline}\label{eq_confuse_with_d_example_2}
    \begin{bmatrix}
    a_1 & a_2 & a_3
    \end{bmatrix}
    \cdot
    \begin{bmatrix}
        c_{11} & c_{12} & c_{13} \\
        c_{12} & c_{22} & c_{23} \\
        c_{13} & c_{23} & c_{33} \\
        \end{bmatrix}
    \cdot
    \begin{bmatrix}
    a_1 \\ a_2 \\ a_3
    \end{bmatrix}
    = c_{11} a_1^2 + c_{22} a_2^2 + c_{33} a_3^2 + {} \\
    {} + 2 c_{12} a_1 a_2 + 2 c_{13} a_1 a_3 + 2 c_{23} a_2 a_3 \ldotp
\end{multline}
Легко видеть, что при выполнении равенств
\begin{gather}
    c_{11} = b_{11},
    \quad
    c_{22} = b_{22},
    \quad
    c_{33} = b_{33},
    \\
    c_{12} = 0.5(b_{12}+b_{21}),
    \quad
    c_{13} = 0.5(b_{13}+b_{31}),
    \quad
    c_{23} = 0.5(b_{23}+b_{32})
\end{gather}
выражения~\eqref{eq_confuse_with_d_example_1} и~\eqref{eq_confuse_with_d_example_2} тоже будут равны.

Подобная ситуация наблюдается и в отношении матриц $\mathcal{D}(q)$ и $D(q)$.