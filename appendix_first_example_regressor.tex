\ESKDappendix{рекомендуемое}{Исключение из регрессора модели плоского двухзвенного манипулятора нулевых и линейно зависимых столбцов}\label{app_first_example_regressor}
Регрессор модели плоского двухзвенного манипулятора равен:
\begin{equation}
    \bar{\xi} =
    \begin{bmatrix}
        \xi_{1,1} & \ddot{q_1} & \dot{q_1} & \sign(\dot{q_1}) & 1 & \xi_{1,2} & 0 & 0 & 0 & 0\\
        O_{1\times10} & 0 & 0 & 0 & 0 & \xi_{1,1} & \ddot{q_2} & \dot{q_2} & \sign(\dot{q_2}) & 1
    \end{bmatrix}\!\!,
\end{equation}
где
\begin{gather}
    \xi_{1,1} =
    \begin{bmatrix}
        \mathcal{L}_1 \{L_{1,1}\} & \mathcal{L}_1 \{L_{1,2}\} & \ldots & \mathcal{L}_1 \{L_{1,10}\}
    \end{bmatrix}\!\!,
    \\
    \xi_{1,2} =
    \begin{bmatrix}
        \mathcal{L}_1 \{L_{2,1}\} & \mathcal{L}_1 \{L_{2,2}\} & \ldots & \mathcal{L}_1 \{L_{2,10}\}
    \end{bmatrix}\!\!,
    \\
    \xi_{2,2} =
    \begin{bmatrix}
        \mathcal{L}_2 \{L_{2,1}\} & \mathcal{L}_2 \{L_{2,2}\} & \ldots & \mathcal{L}_2 \{L_{2,10}\}
    \end{bmatrix}\!\!\ldotp
\end{gather}

После исключения нулевых столбцов, проводимого с учетом уравне-\linebreak %very dirty trick
ний~\eqref{eq_first_regr_comp_for_2}--\eqref{eq_last_regr_comp_for_2}, регрессор $\bar{\xi}$ <<потеряет>> столбцы с номерами, составляющими множество $\{4,5,6,8,9,10,17,19,21,22,23,24\}$.

С~учетом того, что остальные столбцы~$\bar{\xi}$ имеют вид:
\begin{align}
    &\bar{\xi}_{\{1\}} =
    \begin{bmatrix}
        a_1^2 \ddot{q}_1 - a_1 g c_1 \\ 0
    \end{bmatrix}\!\!,
    &
    &\bar{\xi}_{\{11\}} =
    \begin{bmatrix}
        \ddot{q}_1 \\ 0
    \end{bmatrix}\!\!,
    \\
    &\bar{\xi}_{\{2\}} =
    \begin{bmatrix}
        2 a_1 \ddot{q}_1 - g c_1 \\ 0
    \end{bmatrix}\!\!,
    &
    &\bar{\xi}_{\{12\}} =
    \begin{bmatrix}
        \dot{q}_1 \\ 0
    \end{bmatrix}\!\!,
    \\
    &\bar{\xi}_{\{3\}} =
    \begin{bmatrix}
        g s_1 \\ 0
    \end{bmatrix}\!\!,
    &
    &\bar{\xi}_{\{13\}} =
    \begin{bmatrix}
        \sign(\ddot{q}_1) \\ 0
    \end{bmatrix}\!\!,
    \\
    &\bar{\xi}_{\{7\}} =
    \begin{bmatrix}
        \ddot{q}_1 \\ 0
    \end{bmatrix}\!\!,
    &
    &\bar{\xi}_{\{14\}} =
    \begin{bmatrix}
        1 \\ 0
    \end{bmatrix}\!\!,
\end{align}

\begin{align}
    &\bar{\xi}_{\{15\}} =
    \begin{bmatrix}
        \begin{split}
            (a_1^2 + a_2^2 + 2 a_1 a_2 c_2) \ddot{q}_1 &+ (a_2^2 + a_1 a_2 c_2 )\ddot{q}_2 + 2 a_1 a_2 \dot{q}_2 s_2 \dot{q}_1 + {} \\ {} &+ a_1 a_2 \dot{q}_2 s_2 \dot{q}_2 - a_2 g c_{12} - a_1 g c_1
            \\
            (a_2^2 + a_1 a_2 c_2) \ddot{q}_1 &+  a_2^2 \ddot{q}_2 - a_1 a_2 \dot{q}_1 s_2 \dot{q}_1 - a_2 g c_{12}
        \end{split}
    \end{bmatrix}\!\!,
    \\
    &\bar{\xi}_{\{16\}} =
    \begin{bmatrix}
        2 (a_2 + a_1 c_2) \ddot{q}_1 + (2 a_2 + a_1 c_2) \ddot{q}_2 + 2 a_1 \dot{q}_2 s_2 \dot{q}_1 + a_1 \dot{q}_2 s_2 \dot{q}_2 - g c_{12} \\
        (2 a_2 + a_1 c_2) \ddot{q}_1 + 2 a_2 \ddot{q}_2 - a_1 \dot{q}_1 s_2 \dot{q}_1 - g c_{12}
    \end{bmatrix}\!\!,
    \\
    &\bar{\xi}_{\{18\}} =
    \begin{bmatrix}
       2 a_1 s_2 \ddot{q}_1 + a_1 s_2 \ddot{q}_2 - 2 a_1 \dot{q}_2 c_2 \dot{q}_1 - a_1 c_2 \dot{q}_2^2 - g s_{12} \\
       a_1 s_2 \ddot{q}_1 + a_1 \dot{q}_1 c_2 \dot{q}_1 - g s_{12}
    \end{bmatrix}\!\!,
    \\
    &\bar{\xi}_{\{20\}} =
    \begin{bmatrix}
        \ddot{q}_1 + \ddot{q}_2 \\ \ddot{q}_1 + \ddot{q}_2
    \end{bmatrix}\!\!,
    \\
    &\bar{\xi}_{\{25\}} =
    \begin{bmatrix}
        0 \\ \ddot{q}_2
    \end{bmatrix}\!\!,
    \\
    &\bar{\xi}_{\{26\}} =
    \begin{bmatrix}
        0 \\ \dot{q}_2
    \end{bmatrix}\!\!,
    \\
    &\bar{\xi}_{\{27\}} =
    \begin{bmatrix}
        0 \\ \sign(\ddot{q}_2)
    \end{bmatrix}\!\!,
    \\
    &\bar{\xi}_{\{28\}} =
    \begin{bmatrix}
        0 \\ 1
    \end{bmatrix}\!\!,
\end{align}
результаты выполнения алгоритма, описанного в подразделе~\ref{part_lin_depend_columns_deleting}, будут следующими:
\begin{itemize}
    \item в качестве пары линейно зависимых столбцов будут найдены
    \begin{equation}
        \bar{\xi}_{\{7\}} = \bar{\xi}_{\{11\}};
    \end{equation}
    для дальнейшего повествования предположим, что в качестве реакции на этот факт были исключены столбец~$\bar{\xi}_{\{7\}}$ и соответствующий ему параметр~$I_{1,\,zz}$, а параметр~$I_{a,1}$, соответствующий в свою очередь одиннадцатому столбцу, был заменен суммой~$I_{a,1} + I_{1,\,zz}$;
    \item при поиске триплета линейно зависимых столбцов будет определено, что:
    \begin{equation}
        \bar{\xi}_{\{1\}} = a_1 \bar{\xi}_{\{2\}} - a_1^2 \bar{\xi}_{\{11\}};
    \end{equation}
    реакция~--- исключение столбца~$\bar{\xi}_{\{1\}}$, параметра~$m_1$, и следующие замены:
    \begin{gather}
        m_1 x_{c1} \rightarrow m_1 x_{c1} + a_1 m_1, \\
        I_{a,1} + I_{1,\,zz} \rightarrow I_{a,1} + I_{1,\,zz} - a_1^2 m_1;
    \end{gather}
    \item при поиске зависимостей типа <<один столбец от трех других>> никаких результатов найдено не будет;
    \item на последнем этапе алгоритм укажет на то, что
    \begin{equation}
        \bar{\xi}_{\{15\}} = a_2 \bar{\xi}_{\{16\}} - a_2^2 \bar{\xi}_{\{20\}} + a_1 \bar{\xi}_{\{2\}} - a_1^2 \bar{\xi}_{\{11\}};
    \end{equation}
    реакция~--- исключение столбца~$\bar{\xi}_{\{15\}}$, параметра~$m_2$, и следующие замены:
    \begin{gather}
        m_2 x_{c2} \rightarrow m_2 x_{c2} + a_2 m_2, \label{eq_final_change_1}\\
        I_{2,\,yy} \rightarrow I_{2,\,yy} - a_2^2 m_2, \\
        m_1 x_{c1} + a_1 m_1 \rightarrow m_1 x_{c1} + a_1 m_1 + a_1 m_2, \\
        I_{a,1} + I_{1,\,zz} - a_1^2 m_1 \rightarrow I_{a,1} + I_{1,\,zz} - a_1^2 m_1 - a_1^2 m_2 \ldotp \label{eq_final_change_last}
    \end{gather}
\end{itemize}
Иных зависимостей между столбцами регрессора нет.

В~заключение отметим, что при использовании результатов идентификации в каких-либо расчетах, значения, полученные для комбинаций, указанных в правой части выражений~\eqref{eq_final_change_1}--\eqref{eq_final_change_last}, следует приписать соответственно параметрам $m_2 x_{c2}$, $I_{2,\,yy}$, $m_1 x_{c1}$ и $I_{a,1}$, а параметры $m_2$, $m_1$ и $I_{1,\,zz}$ принять равными нулю.
