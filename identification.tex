\section{Идентификация параметров манипулятора}\label{part_identification}
\subsection{Предварительные действия}
\subsubsection{Введение в проблему}
Перед началом описанных ниже экспериментов по определению неизвестных значений параметров робота предварительно необходимо провести два дополнительных действия с его математической моделью, полученной в прошлом разделе.
Суть этих действий описана далее~--- в подразделах~\ref{part_zero_columns_deleting} и~\ref{part_lin_depend_columns_deleting}.


\subsubsection{Исключение нулевых столбцов}~\label{part_zero_columns_deleting}
Первое, что необходимо сделать с моделью робота~\eqref{eq_extended_dynamic_in_linear}~--- это исключить из регрессора~$\bar{\xi}$ все нулевые столбцы, а из вектора параметров~$\bar\chi$~--- соответствующие им параметры.

К примеру\lefteqn,\footnote{Уравнение в примере соответствует случаю $n=2$.} если выражение~\eqref{eq_extended_dynamic_in_linear} получилось имеющим следующий вид (подразумевается, что величины $\xi_{ij}$ таковы, что никакой другой столбик регрессора кроме второго и четвертого не равен нулю):
\begin{equation}
    \underbrace{
        \begin{bmatrix}
            \xi_{11} & 0 & \xi_{13} & 0 & \xi_{15} & \xi_{16} & \xi_{17} & \xi_{18}\\
            \xi_{21} & 0 & \xi_{23} & 0 & \xi_{25} & \xi_{26} & \xi_{27} & \xi_{28}
        \end{bmatrix}
    }_{\displaystyle \bar\xi}
    \cdot
    \underbrace{
        \begin{bmatrix}
            \chi_{11} \\ \chi_{21} \\ \chi_{31} \\ \chi_{41} \\ \chi_{51} \\ \chi_{61} \\ \chi_{71} \\ \chi_{81}
        \end{bmatrix}
    }_{\displaystyle \bar\chi}
    =
    \begin{bmatrix}
        \tau_{e,1} \\ \tau_{e,2}
    \end{bmatrix}\!\!,
\end{equation}
где $\xi_{ij}$ и $\chi_{ij}$~--- элементы регрессора и вектора параметров соответственно, стоящие на пересечении $i$-ой строки и $j$-го столбца, то из регрессора надо исключить второй и четвертый столбик, а из вектора~$\chi$~--- параметры $\chi_{21}$ и $\chi_{41}$.
С~учетом этого исходное уравнение приобретет следующий вид:
\begin{equation}\label{eq_regr_without_zeros}
    \underbrace{
        \begin{bmatrix}
            \xi_{11} & \xi_{13} & \xi_{15} & \xi_{16} & \xi_{17} & \xi_{18}\\
            \xi_{21} & \xi_{23} & \xi_{25} & \xi_{26} & \xi_{27} & \xi_{28}
        \end{bmatrix}
    }_{\displaystyle \bar\xi_0}
    \cdot
    \underbrace{
        \begin{bmatrix}
            \chi_{11} \\ \chi_{31} \\ \chi_{51} \\ \chi_{61} \\ \chi_{71} \\ \chi_{81}
        \end{bmatrix}
    }_{\displaystyle \bar\chi_0}
    =
    \begin{bmatrix}
        \tau_{e,1} \\ \tau_{e,2}
    \end{bmatrix}\!\!\ldotp
\end{equation}

Стоит отметить, что с учетом сложности (громоздкости) выражений для~$\xi_{ij}$ можно не заметить равенство каких-либо из них нулю.
Для исключения такой ситуации рекомендуется подставить в них несколько случайных значений для $q$, $\dot{q}$ и $\ddot{q}$ и посмотреть на получающиеся при этом результаты.


\subsubsection{Исключение линейно зависимых столбцов}~\label{part_lin_depend_columns_deleting}
Второе действие, которое необходимо предпринять по отношению к полученной ранее модели робота~--- это исключить из регрессора~$\bar{\xi}_0$ все линейно зависимые столбцы и соответствующим образом изменить вектор параметров.

К~примеру, пусть относительно уравнения~\eqref{eq_regr_without_zeros} справедливо то, что
\begin{equation}
    \begin{bmatrix}
        \xi_{13} \\ \xi_{23}
    \end{bmatrix}
    =
    k_1
    \begin{bmatrix}
        \xi_{16} \\ \xi_{26}
    \end{bmatrix}\!\!,
    \qquad
    \begin{bmatrix}
        \xi_{11} \\ \xi_{21}
    \end{bmatrix}
    =
    k_2
    \begin{bmatrix}
        \xi_{15} \\ \xi_{25}
    \end{bmatrix}
    +
    k_3
    \begin{bmatrix}
        \xi_{17} \\ \xi_{27}
    \end{bmatrix}\!\!,
\end{equation}
где $k_1$, $k_2$, $k_3$~--- некоторые константы, тогда его надо заменить, например, на такое уравнение:
\begin{equation}
    \underbrace{
        \begin{bmatrix}
            \xi_{15} & \xi_{16} & \xi_{17} & \xi_{18} \\
            \xi_{25} & \xi_{26} & \xi_{27} & \xi_{28}
        \end{bmatrix}
    }_{\displaystyle \check\xi}
    \cdot
    \underbrace{
        \begin{bmatrix}
            k_2\chi_{11} + \chi_{51} \\ k_1\chi_{31} + \chi_{61} \\ k_3\chi_{11} + \chi_{71} \\ \chi_{81}
        \end{bmatrix}
    }_{\displaystyle \check\chi}
    =
    \begin{bmatrix}
        \tau_{e,1} \\ \tau_{e,2}
    \end{bmatrix}\!\!\ldotp
\end{equation}

Для поиска линейно зависимых столбцов авторы использовали следующий алгоритм:
\begin{enumerate}
    \item подстановка $N$ ($N \gg n$) случайных значений для $q$, $\dot{q}$ и $\ddot{q}$ в $\bar{\xi}_0$ и расчет соответствующих реализаций регрессора: $\bar{\xi}_0(1)$, $\bar{\xi}_0(2)$, \ldots, $\bar{\xi}_0(N)$;
    \item формирование новой матрицы $\bar\xi_0^*$, равной
    \begin{equation}
        \bar\xi_0^* =
        \begin{bmatrix}
            \bar{\xi}_0(1) \\ \bar{\xi}_0(2) \\ \vdots \\ \bar{\xi}_0(N)
        \end{bmatrix}\!\!;
    \end{equation}
    \item \label{enum_delete_1}проверка каждого столбца $\bar\xi_0^*$ на зависимость от какого-либо другого ее столбца в виде:
    \begin{equation}
        \bar\xi_0^*{}_{\{i\}} = k \cdot \bar\xi_0^*{}_{\{j\}}, \quad i \ne j,\ k = const
    \end{equation}
    и его удаление по схеме, описанной в данном подразделе выше, в случае, если $k$, удовлетворяющий представленному равенству, существует;
    \item проверка каждого столбца $\bar\xi_0^*$, из которой уже исключены все зависимости, описанные в прошлом пункте, на зависимость от двух других ее столбцов в виде:
    \begin{equation}
        \bar\xi_0^*{}_{\{i\}} = k_1 \cdot \bar\xi_0^*{}_{\{j\}} + k_2 \cdot \bar\xi_0^*{}_{\{k\}}, \quad i \ne j,\ i \ne k,\ j \ne k,\ k_1, k_2 = const
    \end{equation}
    и его удаление по схеме, описанной в данном подразделе выше, в случае, если $k_1$ и $k_2$, удовлетворяющие представленному равенству, существуют;
    \item проверка каждого столбца $\bar\xi_0^*$ на зависимость от трех других ее столбцов и их удаление по методике, аналогичной представленной в двух прошлых пунктах;
    \item \label{enum_delete_many}проверка каждого столбца $\bar\xi_0^*$, в которой, благодаря действиям, описанным в трех прошлых пунктах, нет ни одного квартета, составленного из линейно зависимых столбцов, на зависимость от всех других ее столбцов в виде:
    \begin{equation}
        \bar\xi_0^*{}_{\{i\}} = \sum_{\substack{j\\j\ne i}} k_j \cdot \bar\xi_0^*{}_{\{j\}}, \quad k_j = const;
    \end{equation}
    если данное уравнение имеет единственное решение относительно коэффициентов $k_j$, то соответствующий столбец~$\bar\xi_0^*{}_{\{i\}}$ просто исключается из~$\bar\xi_0^*$ аналогично тому, как это делалось в трех прошлых пунктах; если же данное уравнение имеет бесконечное количество решений, тогда для данного столбца дополнительно выполняется поиск линейной зависимости от четырех, пяти и т.д. столбцов $\bar\xi_0^*$ и его исключение, как только такая зависимость будет найдена.
\end{enumerate}
В~заключение стоит сказать, что каждый из пп.~\ref{enum_delete_1}--\ref{enum_delete_many} стоит повторять до тех пор, пока из регрессора не будут исключены все искомые в рамках данного пункта зависимости.

\subsubsection{Заключительные замечания}
Как можно догадаться, параметры $\chi_{11}$, $\chi_{21}$, $\chi_{31}$, $\chi_{41}$, $\chi_{51}$, $\chi_{61}$ и $\chi_{71}$, затронутые в результате описанных действий, проидентифицировать не получится. При этом, однако, будут найдены значения для следующих комбинаций, включающих некоторые из них: $k_2\chi_{11} + \chi_{51}$, $k_1\chi_{31} + \chi_{61}$ и $k_3\chi_{11} + \chi_{71}$.
В~таких условиях, для того чтобы рассчитывать матрицы, входящие в уравнение~\eqref{eq_model_with_standard_matrix}, или выполнять иные вычисления, необходимо в используемых при этом формулах полагать параметры~$\chi_{11}$, $\chi_{21}$, $\chi_{31}$ и $\chi_{41}$ равными нулю, а в качестве значений для параметров $\chi_{51}$, $\chi_{61}$ и $\chi_{71}$ брать полученные в результате идентификации значения для указанных выше комбинаций соответственно.



\subsection{Описание метода}
Для определения неизвестных значений параметров робота, составляющих вектор $\bar\chi$, воспользуемся методом наименьших квадратов.
Алгоритм необходимых действий в таком случае будет следующим:
\begin{enumerate}
    \item с помощью поставляемого производителем робота ПО\footnote{У~Youbot такое <<стандартное>> ПО осуществляет управление углами в сочленениях робота с помощью ПИД-регуляторов.} дать манипулятору команды на последовательное достижение $N$ произвольных конфигураций, по возможности охватывающих всю его рабочую область; во время его работы снять и записать следующие показания:
        \begin{align*}
            &q(t_1) &&\dot{q}(t_1) &&\ddot{q}(t_1) && \tau_e(t_1) \\
            &q(t_2) &&\dot{q}(t_2) &&\ddot{q}(t_2) && \tau_e(t_2) \\
            &q(t_3) &&\dot{q}(t_3) &&\ddot{q}(t_3) && \tau_e(t_3) \\
            &\ldots &&\ldots &&\ldots && \ldots \\
            &q(t_{k}) &&\dot{q}(t_k) &&\ddot{q}(t_k) && \tau_e(t_k)
        \end{align*}
        где $t_k>t_3>t_2>t_1$;
    \item используя полученные данные, составить матрицы
        \begin{equation}
            \Xi =
            \begin{bmatrix}
                \bar\xi(\ddot{q}(t_1), \dot{q}(t_1), q(t_1)) \\
                \bar\xi(\ddot{q}(t_2), \dot{q}(t_2), q(t_2)) \\
                \ldots \\
                \bar\xi(\ddot{q}(t_k), \dot{q}(t_k), q(t_k))
            \end{bmatrix}\!\!,
            \qquad
            T_e =
            \begin{bmatrix}
                \tau_e(t_1) \\ \tau_e(t_2) \\ \ldots \\ \tau_e(t_k)
            \end{bmatrix}\!\!;
        \end{equation}
    \item \label{item_estimate}рассчитать оценку $\hat{\chi}$ вектора $\bar\chi$ по формуле:
        \begin{equation}
            \hat\chi = ( \Xi^T \!\!\cdot \Xi)^{-1} \cdot \Xi^T \cdot T_e;
        \end{equation}
    \item \label{item_second_data}дать роботу команды на достижение других $N$ позиций и при этом получить те же самые данные;
    \item используя найденную в п.~\ref{item_estimate}) оценку $\hat{\chi}$ и снятые в п.~\ref{item_second_data}) данные, рассчитать по формуле~\eqref{eq_extended_dynamic_in_linear} значения для $\tau_e$; сравнить их c полученными в п.~\ref{item_second_data}) и сделать выводы об успешности идентификации.
\end{enumerate}

\subsection{Результаты}
\newpage
